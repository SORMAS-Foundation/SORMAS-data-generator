% Options for packages loaded elsewhere
\PassOptionsToPackage{unicode}{hyperref}
\PassOptionsToPackage{hyphens}{url}
%
\documentclass[
  8pt,
  ignorenonframetext,
]{beamer}
\usepackage{pgfpages}
\setbeamertemplate{caption}[numbered]
\setbeamertemplate{caption label separator}{: }
\setbeamercolor{caption name}{fg=normal text.fg}
\beamertemplatenavigationsymbolsempty
% Prevent slide breaks in the middle of a paragraph
\widowpenalties 1 10000
\raggedbottom
\setbeamertemplate{part page}{
  \centering
  \begin{beamercolorbox}[sep=16pt,center]{part title}
    \usebeamerfont{part title}\insertpart\par
  \end{beamercolorbox}
}
\setbeamertemplate{section page}{
  \centering
  \begin{beamercolorbox}[sep=12pt,center]{part title}
    \usebeamerfont{section title}\insertsection\par
  \end{beamercolorbox}
}
\setbeamertemplate{subsection page}{
  \centering
  \begin{beamercolorbox}[sep=8pt,center]{part title}
    \usebeamerfont{subsection title}\insertsubsection\par
  \end{beamercolorbox}
}
\AtBeginPart{
  \frame{\partpage}
}
\AtBeginSection{
  \ifbibliography
  \else
    \frame{\sectionpage}
  \fi
}
\AtBeginSubsection{
  \frame{\subsectionpage}
}
\usepackage{lmodern}
\usepackage{amssymb,amsmath}
\usepackage{ifxetex,ifluatex}
\ifnum 0\ifxetex 1\fi\ifluatex 1\fi=0 % if pdftex
  \usepackage[T1]{fontenc}
  \usepackage[utf8]{inputenc}
  \usepackage{textcomp} % provide euro and other symbols
\else % if luatex or xetex
  \usepackage{unicode-math}
  \defaultfontfeatures{Scale=MatchLowercase}
  \defaultfontfeatures[\rmfamily]{Ligatures=TeX,Scale=1}
\fi
% Use upquote if available, for straight quotes in verbatim environments
\IfFileExists{upquote.sty}{\usepackage{upquote}}{}
\IfFileExists{microtype.sty}{% use microtype if available
  \usepackage[]{microtype}
  \UseMicrotypeSet[protrusion]{basicmath} % disable protrusion for tt fonts
}{}
\makeatletter
\@ifundefined{KOMAClassName}{% if non-KOMA class
  \IfFileExists{parskip.sty}{%
    \usepackage{parskip}
  }{% else
    \setlength{\parindent}{0pt}
    \setlength{\parskip}{6pt plus 2pt minus 1pt}}
}{% if KOMA class
  \KOMAoptions{parskip=half}}
\makeatother
\usepackage{xcolor}
\IfFileExists{xurl.sty}{\usepackage{xurl}}{} % add URL line breaks if available
\IfFileExists{bookmark.sty}{\usepackage{bookmark}}{\usepackage{hyperref}}
\hypersetup{
  pdftitle={Generation of a credible test data set for SORMAS},
  pdfauthor={Stéphane Ghozzi Hemholtz Centre for Infection Research (HZI) stephane.ghozzi@helmholtz-hzi.de },
  hidelinks,
  pdfcreator={LaTeX via pandoc}}
\urlstyle{same} % disable monospaced font for URLs
\newif\ifbibliography
\setlength{\emergencystretch}{3em} % prevent overfull lines
\providecommand{\tightlist}{%
  \setlength{\itemsep}{0pt}\setlength{\parskip}{0pt}}
\setcounter{secnumdepth}{-\maxdimen} % remove section numbering
\usepackage{graphicx}
\usepackage[T1]{fontenc}
\usepackage[USenglish]{babel}
\usepackage{hyperref}
\usepackage{url}
\usepackage{cite}
\usepackage{amssymb}
\usepackage{amsmath}
\usepackage{bm}
\usepackage{tikz}

\setbeamertemplate{navigation symbols}{}
\defbeamertemplate{footline}{title and page number}{%
    \color{gray!95}
    \hspace{2em} \inserttitle \hfill%
    \insertpagenumber\,/\,\insertpresentationendpage\kern2em\vskip2pt%
    \vspace{0.2cm}%
}
\setbeamertemplate{footline}[title and page number]{}
\setbeamertemplate{frametitle}{
    \vspace{0.5cm}
    \insertframetitle
}
\setlength{\fboxsep}{0pt}
\setlength{\fboxrule}{0.1pt}
\useinnertheme{circles}
\usefonttheme{serif}

\definecolor{vegablue}{HTML}{1F77B4}
\definecolor{vegaorange}{HTML}{FF7F0E}
\definecolor{vegagreen}{HTML}{2CA02C}
\setbeamercolor{title}{fg=black}
\setbeamercolor{subtitle}{fg=vegablue}
\setbeamercolor{frametitle}{fg=vegablue}
\setbeamercolor{structure}{fg=black}

\hypersetup{colorlinks=true,linkcolor=vegablue,urlcolor=vegablue,citecolor=vegablue,anchorcolor=vegablue}

\newcommand{\p}{\mathrm{p}}
\newcommand{\x}{\bm{x}}
\ifluatex
  \usepackage{selnolig}  % disable illegal ligatures
\fi

\title{Generation of a credible test data set for SORMAS}
\subtitle{HZI -- 4 November 2020}
\author{\newline \newline Stéphane Ghozzi \newline \small Hemholtz
Centre for Infection Research (HZI)
\newline \href{mailto:stephane.ghozzi@helmholtz-hzi.de}{\nolinkurl{stephane.ghozzi@helmholtz-hzi.de}}
\normalsize}
\date{}

\begin{document}
\frame{\titlepage}

\begin{frame}{Approach}
\protect\hypertarget{approach}{}
COVID-19 case counts from RKI for Braunschweig, Salzgitter, Wolfsburg:

\begin{itemize}
\tightlist
\item
  corona dashboard: per county, reporting day, onset day, age group, sex
\item
  SurvStat: per county, reporting week, case definition
\end{itemize}

\vspace{1.5cm}

Generate individual \textbf{cases} that reproduce these counts
\end{frame}

\begin{frame}
Add \textbf{case features}

\begin{itemize}
\tightlist
\item
  onset date (Gumbel distribution fitted to data)
\item
  infection date (\textasciitilde{} exp, scale = 3 days)
\item
  country of residence (Germany/France, 5\%)
\item
  first and family names (``German + \textasciitilde Turkish +
  \textasciitilde Polish'' / ``French'')
\item
  address \textasciitilde{} county of reporting LHA (85\%)
\item
  hospitalization (10\%)
\item
  death (3\%)
\item
  symptoms \textasciitilde{} case definition + list (rough probas,
  incl.~asymptomatic)
\end{itemize}
\end{frame}

\begin{frame}
Create chains of \textbf{infections}

\begin{itemize}
\tightlist
\item
  cases of generation 0, 1, 2, 3, 4 (e.g.~0 = 20\%)
\item
  added sequentially
\item
  p(infection) \textasciitilde{} degree, exp(diff infection date)
\item
  location \textasciitilde{} close to index (exp), gen 0: random in
  address
\end{itemize}
\end{frame}

\begin{frame}
Create \textbf{contacts}

\begin{itemize}
\tightlist
\item
  number of contacts \textasciitilde{} social contact matrix / age
  (POLYMOD)
\item
  location \textasciitilde{} close to index (exp)
\item
  address \textasciitilde{} address index (67\%)
\item
  contacted (50\%)
\item
  quarantine (75\%)
\item
  tested \textbar{} contacted (75\%)
\end{itemize}

\vspace{0.5cm}

contacts are not shared between cases
\end{frame}

\begin{frame}
Create \textbf{events} and \textbf{participants}

\begin{itemize}
\tightlist
\item
  n participants \textasciitilde{} N(10, 10), at least 5
\item
  at least 1 case
\item
  event date \textasciitilde{} w/in 2 weeks after infection date of case
\item
  1\% chance that a participant is a case or a contact
\item
  participants: random locations
\item
  event: location = center of mass
\end{itemize}
\end{frame}

\begin{frame}{Times series}
\protect\hypertarget{times-series}{}
\begin{center}
\includegraphics[width=0.8\textwidth]{../img/case_count_ts_plot.pdf}
\end{center}
\end{frame}

\begin{frame}
\begin{center}
\includegraphics[width=0.8\textwidth]{../img/Rt_plot.pdf}
\end{center}
\end{frame}

\begin{frame}
\begin{center}
\includegraphics[width=0.8\textwidth]{../img/Dt_plot.pdf}
\end{center}
\end{frame}

\begin{frame}
\begin{center}
\includegraphics[width=0.8\textwidth]{../img/anomaly_detection_plot.pdf}
\end{center}
\end{frame}

\begin{frame}
Nowcasting and \(R_e(t)\)

from

Felix Günther, Andreas Bender, Katharina Katz, Helmut Küchenhoff,
Michael Höhle \newline  Nowcasting the COVID-19 pandemic in Bavaria
\newline
\url{https://www.medrxiv.org/content/10.1101/2020.06.26.20140210v2}
\newline to be published in Biometrical Journal

\url{https://github.com/FelixGuenther/nc_covid19_bavaria}

\vspace{1cm}

\begin{center}
coming soon!
\end{center}
\end{frame}

\begin{frame}{Distributions}
\protect\hypertarget{distributions}{}
\begin{center}
\includegraphics[width=0.8\textwidth]{../img/reporting_delays_plot.pdf}
\end{center}
\end{frame}

\begin{frame}
\begin{center}
\includegraphics[width=0.8\textwidth]{../img/infect_intervals_plot.pdf}
\end{center}
\end{frame}

\begin{frame}
\begin{center}
\includegraphics[width=0.8\textwidth]{../img/marginal_symptoms_plot.pdf}
\end{center}
\end{frame}

\begin{frame}
\begin{center}
\includegraphics[width=0.8\textwidth]{../img/joint_symptoms_plot.pdf}
\end{center}
\end{frame}

\begin{frame}{Graph}
\protect\hypertarget{graph}{}
building the complete graph \textasciitilde{} 8 s

\begin{center}
\includegraphics[width=0.8\textwidth]{../img/cases_graph.pdf}
\end{center}
\end{frame}

\begin{frame}
\begin{center}
\includegraphics[width=0.8\textwidth]{../img/cases_contacts_graph_plot.pdf}
\end{center}
\end{frame}

\begin{frame}
\begin{center}
\includegraphics[width=0.8\textwidth]{../img/components/all/abstract_graph.pdf}
\end{center}
\end{frame}

\begin{frame}
\begin{center}
\includegraphics[width=0.8\textwidth]{../img/components/all/geo_graph_map.pdf}
\end{center}
\end{frame}

\begin{frame}
\begin{center}
\includegraphics[width=0.8\textwidth]{../img/degrees_plot.pdf}
\end{center}
\end{frame}

\begin{frame}{Filter graph by infection date}
\protect\hypertarget{filter-graph-by-infection-date}{}
as an example\ldots{} \emph{reporting} date might be more relevant

\begin{center}
\includegraphics[width=0.8\textwidth]{../img/graph_infectdate-2020-03-20.pdf}
\end{center}
\end{frame}

\begin{frame}{Example 1: component with infection chain}
\protect\hypertarget{example-1-component-with-infection-chain}{}
abstract visualization

\begin{center}
\includegraphics[width=0.8\textwidth]{../img/components/14/abstract_graph.pdf}
\end{center}
\end{frame}

\begin{frame}
static visualization in space

\begin{center}
\includegraphics[width=0.8\textwidth]{../img/components/14/geo_graph_map.pdf}
\end{center}
\end{frame}

\begin{frame}
interactive visualization in space

\begin{center}
\includegraphics[width=0.8\textwidth]{../img/components/14/subnetwork_component.png}
\end{center}
\end{frame}

\begin{frame}{Example 2: component with event}
\protect\hypertarget{example-2-component-with-event}{}
abstract visualization

\begin{center}
\includegraphics[width=0.8\textwidth]{../img/components/55/abstract_graph.pdf}
\end{center}
\end{frame}

\begin{frame}
static visualization in space

\begin{center}
\includegraphics[width=0.8\textwidth]{../img/components/55/geo_graph_map.pdf}
\end{center}
\end{frame}

\begin{frame}
interactive visualization in space

\begin{center}
\includegraphics[width=0.8\textwidth]{../img/components/55/subnetwork_component.png}
\end{center}
\end{frame}

\begin{frame}{Component features}
\protect\hypertarget{component-features}{}
\begin{center}
\includegraphics[width=0.8\textwidth]{../img/csize_plot.pdf}
\end{center}

\tiny

overall: 353 components \normalsize
\end{frame}

\begin{frame}
\begin{center}
\includegraphics[width=0.8\textwidth]{../img/longest_path_relativesizes_plot.pdf}
\end{center}
\end{frame}

\begin{frame}
\begin{center}
\includegraphics[width=0.8\textwidth]{../img/shortest_path_plot.pdf}
\end{center}
\end{frame}

\begin{frame}{Inter-regional graphs}
\protect\hypertarget{inter-regional-graphs}{}
Nodes in county, size = \# persons

Edges between counties, width = \# contacts or participations

\vspace{1cm}

\begin{center}
coming soon!
\end{center}
\end{frame}

\begin{frame}{Outlook}
\protect\hypertarget{outlook}{}
\textbf{Interface} to SORMAS and SORMAS-Stats formats:

\begin{itemize}
\tightlist
\item
  export test data to SORMAS
\item
  import SORMAS data to run code (vis + analysis)
\item
  export results to SORMAS-Stats
\end{itemize}

\vspace{1cm}

in progress: Jonas + export from SORMAS test server
\end{frame}

\begin{frame}
Improve the \textbf{generation process:}

\begin{itemize}
\tightlist
\item
  reduce \textbf{data quality} (add missing, incorrect values)
\item
  statistics-based and age-dependent prob. hospitalization, prob. death
\item
  age-dependent infectiosity
\item
  age distribution in age group
\item
  prob. test \textasciitilde{} testing rate
\item
  non-tree structure (already loops possible through events)
\item
  events are local
\end{itemize}
\end{frame}

\begin{frame}
Add \textbf{features:}

\begin{itemize}
\tightlist
\item
  proper address \textasciitilde{} location / phone number / birth date
  / sex ``divers''
\item
  time and result of test
\item
  isolation/quarantine dates
\item
  hospitalization and death dates
\item
  ICU
\item
  \textbf{time dynamics of LHA workflow,} e.g., date of contact, of
  quarantine
\item
  professional occupation of case
\item
  \textbf{setting/context of infection, of event:} type, name
\end{itemize}
\end{frame}

\begin{frame}
\textbf{Visualizations} and \textbf{analyses:}

\begin{itemize}
\tightlist
\item
  nowcast and \(R_e(t)\)
\item
  aggregation over regions
\item
  \textbf{prioritization} (backward tracing), cf.~\emph{betweenness}
\item
  aberration detection vs.~components
\item
  sizes/number of components in time
\item
  cumulative number of cases, contacts, participants
\item
  dispersion K
\item
  modeling SEIR, under-reporting
\end{itemize}
\end{frame}

\begin{frame}[fragile]{Data structure}
\protect\hypertarget{data-structure}{}
\texttt{persons\_df}

\vspace{0.2cm}

\tiny

\begin{verbatim}
tibble [12,867 × 29] (S3: tbl_df/tbl/data.frame)
 $ local_health_authority: chr [1:12867] "Braunschweig" "Braunschweig" "Braunschweig" "Braunschweig" ...
 $ sex                   : chr [1:12867] "w" "w" "w" "w" ...
 $ age_group             : chr [1:12867] "A00-A04" "A00-A04" "A00-A04" "A00-A04" ...
 $ reporting_date        : Date[1:12867], format: "2020-03-22" "2020-03-31" "2020-09-02" "2020-10-06" ...
 $ reporting_week        : chr [1:12867] "2020-W12" "2020-W14" "2020-W36" "2020-W41" ...
 $ onset_date            : Date[1:12867], format: "2020-03-17" "2020-03-30" "2020-09-06" "2020-09-28" ...
 $ id                    : num [1:12867] 1 2 3 4 5 6 7 8 9 10 ...
 $ is_case               : logi [1:12867] TRUE TRUE TRUE TRUE TRUE TRUE ...
 $ age                   : int [1:12867] 4 3 1 0 0 0 11 14 12 13 ...
 $ country_of_residence  : chr [1:12867] "Deutschland" "Deutschland" "Deutschland" "Deutschland" ...
 $ address               : chr [1:12867] "Braunschweig" "Braunschweig" "Wolfsburg" "Braunschweig" ...
 $ infection_date        : Date[1:12867], format: "2020-03-17" "2020-03-28" "2020-09-01" "2020-09-28" ...
 $ case_def_id           : chr [1:12867] "D" "C" "E" "E" ...
 $ case_def              : chr [1:12867] "labordiagnostisch bei nicht erfüllter Klinik" "klinisch-labordiagnostisch" "labordiagnostisch bei unbekannter Klinik" "labordiagnostisch bei unbekannter Klinik" ...
 $ has_symptoms          : logi [1:12867] FALSE TRUE TRUE TRUE TRUE TRUE ...
 $ hospitalized          : logi [1:12867] FALSE FALSE TRUE TRUE TRUE TRUE ...
 $ died                  : logi [1:12867] FALSE FALSE FALSE FALSE FALSE FALSE ...
 $ generation            : chr [1:12867] "4" "4" "3" "0" ...
 $ infected_by           : int [1:12867] 117 342 289 NA 1098 NA NA 1106 NA NA ...
 $ degree                : num [1:12867] 0 0 0 0 0 1 2 0 0 1 ...
 $ has_contact           : logi [1:12867] TRUE TRUE TRUE FALSE TRUE FALSE ...
 $ contact_contacted     : logi [1:12867] NA NA NA NA NA NA ...
 $ contact_quarantine    : logi [1:12867] NA NA NA NA NA NA ...
 $ contact_tested        : logi [1:12867] NA NA NA NA NA NA ...
 $ first_name            : chr [1:12867] "Andrea" "Andrea" "Sabine" "Monika" ...
 $ family_name           : chr [1:12867] "Nowak" "Braun" "Koch" "Yılmaz" ...
 $ longitude             : num [1:12867] 10.6 10.6 10.7 10.5 10.6 ...
 $ latitude              : num [1:12867] 52.3 52.3 52.4 52.3 52.2 ...
 $ was_in_event          : logi [1:12867] FALSE FALSE FALSE FALSE FALSE FALSE ...
\end{verbatim}

\normalsize
\end{frame}

\begin{frame}[fragile]
\texttt{symptoms\_cases\_df}

\vspace{0.2cm}

\tiny

\begin{verbatim}
# A tibble: 3,217 x 3
      id symptom             id_person
   <dbl> <chr>                   <int>
 1     1 Fieber                      2
 2     2 Geruchssinnsverlust         2
 3     3 Niesen                      2
 4     4 Kopfschmerzen               2
 5     5 Atemnot                     2
 6     6 Husten                      3
 7     7 Geruchssinnsverlust         3
 8     8 Schmerzen                   3
 9     9 Halssschmerz                3
10    10 Kopfschmerzen               3
# … with 3,207 more rows
\end{verbatim}

\normalsize
\end{frame}

\begin{frame}[fragile]
\texttt{contacts\_df}

\vspace{0.2cm}

\tiny

\begin{verbatim}
# A tibble: 11,589 x 4
      id id_index id_contact is_infection
   <dbl>    <int>      <dbl> <lgl>       
 1     1      117          1 TRUE        
 2     2      342          2 TRUE        
 3     3      289          3 TRUE        
 4     4     1098          5 TRUE        
 5     5     1106          8 TRUE        
 6     6      333         15 TRUE        
 7     7     1369         16 TRUE        
 8     8      897         17 TRUE        
 9     9      185         18 TRUE        
10    10       22         19 TRUE        
# … with 11,579 more rows
\end{verbatim}

\normalsize
\end{frame}

\begin{frame}[fragile]
\texttt{events\_df}

\vspace{0.2cm}

\tiny

\begin{verbatim}
# A tibble: 69 x 5
      id date       address      longitude latitude
   <int> <date>     <chr>            <dbl>    <dbl>
 1     1 2020-10-17 Braunschweig      10.6     52.3
 2     2 2020-10-05 Braunschweig      10.6     52.3
 3     3 2020-09-19 Braunschweig      10.5     52.2
 4     4 2020-03-13 Braunschweig      10.6     52.3
 5     5 2020-08-07 Braunschweig      10.5     52.3
 6     6 2020-04-09 Braunschweig      10.6     52.3
 7     7 2020-08-04 Braunschweig      10.6     52.3
 8     8 2020-04-18 Braunschweig      10.6     52.3
 9     9 2020-03-31 Braunschweig      10.6     52.3
10    10 2020-08-06 Braunschweig      10.6     52.3
# … with 59 more rows
\end{verbatim}

\normalsize
\end{frame}

\begin{frame}[fragile]
\texttt{event\_participants\_df}

\vspace{0.2cm}

\tiny

\begin{verbatim}
# A tibble: 994 x 3
      id id_event id_participant
   <dbl>    <int>          <dbl>
 1     1        1           1243
 2     2        1          11950
 3     3        1          11951
 4     4        1          11952
 5     5        1          11953
 6     6        1          11954
 7     7        1          11955
 8     8        1          11956
 9     9        1          11957
10    10        1          11958
# … with 984 more rows
\end{verbatim}

\normalsize
\end{frame}

\begin{frame}[fragile]
\texttt{network\_graph}

\vspace{0.2cm}

\tiny

\begin{verbatim}
# A tbl_graph: 12936 nodes and 12583 edges
#
# A rooted forest with 353 trees
#
# Node Data: 12,936 x 6 (active)
  id    type  root_case date                  geometry degree
  <chr> <chr> <lgl>     <date>                 <POINT>  <dbl>
1 1     case  FALSE     2020-03-17 (10.62963 52.29662)     12
2 2     case  FALSE     2020-03-28  (10.57941 52.2834)      8
3 3     case  FALSE     2020-09-01 (10.70493 52.39894)      7
4 4     case  TRUE      2020-09-28  (10.46876 52.2861)      4
5 5     case  FALSE     2020-10-03  (10.5652 52.24183)      4
6 6     case  TRUE      2020-09-08 (10.48293 52.27645)      6
# … with 12,930 more rows
#
# Edge Data: 12,583 x 3
   from    to                               geometry
  <int> <int>                           <LINESTRING>
1   117     1 (10.69714 52.37936, 10.62963 52.29662)
2   342     2  (10.61909 52.29242, 10.57941 52.2834)
3   289     3 (10.55414 52.27636, 10.70493 52.39894)
# … with 12,580 more rows
\end{verbatim}

\normalsize
\end{frame}

\end{document}
